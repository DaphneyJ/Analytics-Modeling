% Options for packages loaded elsewhere
\PassOptionsToPackage{unicode}{hyperref}
\PassOptionsToPackage{hyphens}{url}
%
\documentclass[
]{article}
\usepackage{amsmath,amssymb}
\usepackage{iftex}
\ifPDFTeX
  \usepackage[T1]{fontenc}
  \usepackage[utf8]{inputenc}
  \usepackage{textcomp} % provide euro and other symbols
\else % if luatex or xetex
  \usepackage{unicode-math} % this also loads fontspec
  \defaultfontfeatures{Scale=MatchLowercase}
  \defaultfontfeatures[\rmfamily]{Ligatures=TeX,Scale=1}
\fi
\usepackage{lmodern}
\ifPDFTeX\else
  % xetex/luatex font selection
\fi
% Use upquote if available, for straight quotes in verbatim environments
\IfFileExists{upquote.sty}{\usepackage{upquote}}{}
\IfFileExists{microtype.sty}{% use microtype if available
  \usepackage[]{microtype}
  \UseMicrotypeSet[protrusion]{basicmath} % disable protrusion for tt fonts
}{}
\makeatletter
\@ifundefined{KOMAClassName}{% if non-KOMA class
  \IfFileExists{parskip.sty}{%
    \usepackage{parskip}
  }{% else
    \setlength{\parindent}{0pt}
    \setlength{\parskip}{6pt plus 2pt minus 1pt}}
}{% if KOMA class
  \KOMAoptions{parskip=half}}
\makeatother
\usepackage{xcolor}
\usepackage[margin=1in]{geometry}
\usepackage{color}
\usepackage{fancyvrb}
\newcommand{\VerbBar}{|}
\newcommand{\VERB}{\Verb[commandchars=\\\{\}]}
\DefineVerbatimEnvironment{Highlighting}{Verbatim}{commandchars=\\\{\}}
% Add ',fontsize=\small' for more characters per line
\usepackage{framed}
\definecolor{shadecolor}{RGB}{248,248,248}
\newenvironment{Shaded}{\begin{snugshade}}{\end{snugshade}}
\newcommand{\AlertTok}[1]{\textcolor[rgb]{0.94,0.16,0.16}{#1}}
\newcommand{\AnnotationTok}[1]{\textcolor[rgb]{0.56,0.35,0.01}{\textbf{\textit{#1}}}}
\newcommand{\AttributeTok}[1]{\textcolor[rgb]{0.13,0.29,0.53}{#1}}
\newcommand{\BaseNTok}[1]{\textcolor[rgb]{0.00,0.00,0.81}{#1}}
\newcommand{\BuiltInTok}[1]{#1}
\newcommand{\CharTok}[1]{\textcolor[rgb]{0.31,0.60,0.02}{#1}}
\newcommand{\CommentTok}[1]{\textcolor[rgb]{0.56,0.35,0.01}{\textit{#1}}}
\newcommand{\CommentVarTok}[1]{\textcolor[rgb]{0.56,0.35,0.01}{\textbf{\textit{#1}}}}
\newcommand{\ConstantTok}[1]{\textcolor[rgb]{0.56,0.35,0.01}{#1}}
\newcommand{\ControlFlowTok}[1]{\textcolor[rgb]{0.13,0.29,0.53}{\textbf{#1}}}
\newcommand{\DataTypeTok}[1]{\textcolor[rgb]{0.13,0.29,0.53}{#1}}
\newcommand{\DecValTok}[1]{\textcolor[rgb]{0.00,0.00,0.81}{#1}}
\newcommand{\DocumentationTok}[1]{\textcolor[rgb]{0.56,0.35,0.01}{\textbf{\textit{#1}}}}
\newcommand{\ErrorTok}[1]{\textcolor[rgb]{0.64,0.00,0.00}{\textbf{#1}}}
\newcommand{\ExtensionTok}[1]{#1}
\newcommand{\FloatTok}[1]{\textcolor[rgb]{0.00,0.00,0.81}{#1}}
\newcommand{\FunctionTok}[1]{\textcolor[rgb]{0.13,0.29,0.53}{\textbf{#1}}}
\newcommand{\ImportTok}[1]{#1}
\newcommand{\InformationTok}[1]{\textcolor[rgb]{0.56,0.35,0.01}{\textbf{\textit{#1}}}}
\newcommand{\KeywordTok}[1]{\textcolor[rgb]{0.13,0.29,0.53}{\textbf{#1}}}
\newcommand{\NormalTok}[1]{#1}
\newcommand{\OperatorTok}[1]{\textcolor[rgb]{0.81,0.36,0.00}{\textbf{#1}}}
\newcommand{\OtherTok}[1]{\textcolor[rgb]{0.56,0.35,0.01}{#1}}
\newcommand{\PreprocessorTok}[1]{\textcolor[rgb]{0.56,0.35,0.01}{\textit{#1}}}
\newcommand{\RegionMarkerTok}[1]{#1}
\newcommand{\SpecialCharTok}[1]{\textcolor[rgb]{0.81,0.36,0.00}{\textbf{#1}}}
\newcommand{\SpecialStringTok}[1]{\textcolor[rgb]{0.31,0.60,0.02}{#1}}
\newcommand{\StringTok}[1]{\textcolor[rgb]{0.31,0.60,0.02}{#1}}
\newcommand{\VariableTok}[1]{\textcolor[rgb]{0.00,0.00,0.00}{#1}}
\newcommand{\VerbatimStringTok}[1]{\textcolor[rgb]{0.31,0.60,0.02}{#1}}
\newcommand{\WarningTok}[1]{\textcolor[rgb]{0.56,0.35,0.01}{\textbf{\textit{#1}}}}
\usepackage{graphicx}
\makeatletter
\def\maxwidth{\ifdim\Gin@nat@width>\linewidth\linewidth\else\Gin@nat@width\fi}
\def\maxheight{\ifdim\Gin@nat@height>\textheight\textheight\else\Gin@nat@height\fi}
\makeatother
% Scale images if necessary, so that they will not overflow the page
% margins by default, and it is still possible to overwrite the defaults
% using explicit options in \includegraphics[width, height, ...]{}
\setkeys{Gin}{width=\maxwidth,height=\maxheight,keepaspectratio}
% Set default figure placement to htbp
\makeatletter
\def\fps@figure{htbp}
\makeatother
\setlength{\emergencystretch}{3em} % prevent overfull lines
\providecommand{\tightlist}{%
  \setlength{\itemsep}{0pt}\setlength{\parskip}{0pt}}
\setcounter{secnumdepth}{-\maxdimen} % remove section numbering
\ifLuaTeX
  \usepackage{selnolig}  % disable illegal ligatures
\fi
\IfFileExists{bookmark.sty}{\usepackage{bookmark}}{\usepackage{hyperref}}
\IfFileExists{xurl.sty}{\usepackage{xurl}}{} % add URL line breaks if available
\urlstyle{same}
\hypersetup{
  pdftitle={ISYE HW1},
  hidelinks,
  pdfcreator={LaTeX via pandoc}}

\title{ISYE HW1}
\author{}
\date{\vspace{-2.5em}2024-01-16}

\begin{document}
\maketitle

\textbf{Question 2.1}

\emph{Describe a situation or problem from your job, everyday life,
current events, etc., for which a classification model would be
appropriate. List some (up to 5) predictors that you might use.}

I could use the classification model to optimize my rental business by
predicting the occupancy of the properties during a specific time
period. The informed decisions could help ensure that the properties are
marketed effectively, priced competitively, and occupied at optimal
levels throughout the year. Predictors: Pricing trends Seasonal trends
Local events Historical booking history

\textbf{Question 2.2}

\emph{The files credit\_card\_data.txt (without headers) and
credit\_card\_data-headers.txt (with headers) contain a dataset with 654
data points, 6 continuous and 4 binary predictor variables. It has
anonymized credit card applications with a binary response variable
(last column) indicating if the application was positive or negative.
The dataset is the ``Credit Approval Data Set'' from the UCI Machine
Learning Repository
(\url{https://archive.ics.uci.edu/ml/datasets/Credit+Approval}) without
the categorical variables and without data points that have missing
values.}

\emph{1. Using the support vector machine function ksvm contained in the
R package kernlab, find a good classifier for this data. Show the
equation of your classifier, and how well it classifies the data points
in the full data set. (Don't worry about test/validation data yet; we'll
cover that topic soon.)}

\begin{Shaded}
\begin{Highlighting}[]
\CommentTok{\#Load the data}
\NormalTok{data }\OtherTok{\textless{}{-}} \FunctionTok{read.table}\NormalTok{(}\StringTok{"credit\_card\_data{-}headers.txt"}\NormalTok{, }\AttributeTok{header =} \ConstantTok{TRUE}\NormalTok{)}
\end{Highlighting}
\end{Shaded}

\begin{Shaded}
\begin{Highlighting}[]
\CommentTok{\#Look at the data}
\FunctionTok{head}\NormalTok{(data) }
\end{Highlighting}
\end{Shaded}

\begin{verbatim}
##   A1    A2    A3   A8 A9 A10 A11 A12 A14 A15 R1
## 1  1 30.83 0.000 1.25  1   0   1   1 202   0  1
## 2  0 58.67 4.460 3.04  1   0   6   1  43 560  1
## 3  0 24.50 0.500 1.50  1   1   0   1 280 824  1
## 4  1 27.83 1.540 3.75  1   0   5   0 100   3  1
## 5  1 20.17 5.625 1.71  1   1   0   1 120   0  1
## 6  1 32.08 4.000 2.50  1   1   0   0 360   0  1
\end{verbatim}

\begin{Shaded}
\begin{Highlighting}[]
\FunctionTok{tail}\NormalTok{(data)}
\end{Highlighting}
\end{Shaded}

\begin{verbatim}
##     A1    A2     A3   A8 A9 A10 A11 A12 A14 A15 R1
## 649  1 40.58  3.290 3.50  0   1   0   0 400   0  0
## 650  1 21.08 10.085 1.25  0   1   0   1 260   0  0
## 651  0 22.67  0.750 2.00  0   0   2   0 200 394  0
## 652  0 25.25 13.500 2.00  0   0   1   0 200   1  0
## 653  1 17.92  0.205 0.04  0   1   0   1 280 750  0
## 654  1 35.00  3.375 8.29  0   1   0   0   0   0  0
\end{verbatim}

\begin{Shaded}
\begin{Highlighting}[]
\CommentTok{\#Load the package kernlab which contains ksvm}
\FunctionTok{library}\NormalTok{(kernlab)}
\end{Highlighting}
\end{Shaded}

\begin{Shaded}
\begin{Highlighting}[]
\CommentTok{\#Run the model; use the ksvm function with simple linear kernel Vanilladot}
\CommentTok{\#convert to matrix format }
\NormalTok{model1 }\OtherTok{\textless{}{-}} \FunctionTok{ksvm}\NormalTok{(}\FunctionTok{as.matrix}\NormalTok{(data[,}\DecValTok{1}\SpecialCharTok{:}\DecValTok{10}\NormalTok{]),}\FunctionTok{as.factor}\NormalTok{(data[,}\DecValTok{11}\NormalTok{]), }\AttributeTok{C =} \DecValTok{100}\NormalTok{, }\AttributeTok{scaled =} \ConstantTok{TRUE}\NormalTok{, }\AttributeTok{kernel =} \StringTok{"vanilladot"}\NormalTok{, }\AttributeTok{type =} \StringTok{"C{-}svc"}\NormalTok{)}
\end{Highlighting}
\end{Shaded}

\begin{verbatim}
##  Setting default kernel parameters
\end{verbatim}

\begin{Shaded}
\begin{Highlighting}[]
\CommentTok{\#Calculate coefficients }
\NormalTok{a }\OtherTok{\textless{}{-}} \FunctionTok{colSums}\NormalTok{(model1}\SpecialCharTok{@}\NormalTok{xmatrix[[}\DecValTok{1}\NormalTok{]]}\SpecialCharTok{*}\NormalTok{ model1}\SpecialCharTok{@}\NormalTok{coef[[}\DecValTok{1}\NormalTok{]])}
\FunctionTok{print}\NormalTok{(a)}
\end{Highlighting}
\end{Shaded}

\begin{verbatim}
##            A1            A2            A3            A8            A9 
## -0.0010065348 -0.0011729048 -0.0016261967  0.0030064203  1.0049405641 
##           A10           A11           A12           A14           A15 
## -0.0028259432  0.0002600295 -0.0005349551 -0.0012283758  0.1063633995
\end{verbatim}

\begin{Shaded}
\begin{Highlighting}[]
\CommentTok{\#Calculate a0}
\NormalTok{a0 }\OtherTok{\textless{}{-}} \SpecialCharTok{{-}}\NormalTok{model1}\SpecialCharTok{@}\NormalTok{b}
\FunctionTok{print}\NormalTok{(a0)}
\end{Highlighting}
\end{Shaded}

\begin{verbatim}
## [1] 0.08158492
\end{verbatim}

\begin{Shaded}
\begin{Highlighting}[]
\CommentTok{\#Show the model predictions }
\NormalTok{pred }\OtherTok{\textless{}{-}} \FunctionTok{predict}\NormalTok{(model1,data[,}\DecValTok{1}\SpecialCharTok{:}\DecValTok{10}\NormalTok{])}
\end{Highlighting}
\end{Shaded}

\begin{Shaded}
\begin{Highlighting}[]
\CommentTok{\#Test the accuracy of the model’s predictions}
\NormalTok{accuracy }\OtherTok{\textless{}{-}} \FunctionTok{sum}\NormalTok{(pred }\SpecialCharTok{==}\NormalTok{ data[,}\DecValTok{11}\NormalTok{])}\SpecialCharTok{/}\FunctionTok{nrow}\NormalTok{(data)}\SpecialCharTok{*} \DecValTok{100}
\FunctionTok{print}\NormalTok{(accuracy)}
\end{Highlighting}
\end{Shaded}

\begin{verbatim}
## [1] 86.39144
\end{verbatim}

\begin{Shaded}
\begin{Highlighting}[]
\CommentTok{\# 0.86391.. {-}\textgreater{} 86.391\%, This means the models’ accuracy is 86.391\% }

\CommentTok{\#I calculated the accuracy at different C values and found that adjusting the value of C did not change the outcome of the model. }

\CommentTok{\#the classifier’s equation: {-}0.001A1 {-} 0.00117A2 {-} 0.0016A3 + 0.003A8 + 1.0049A9 {-} 0.0028A10 + 0.00026A11 {-} 0.0005A12 {-} 0.0012A14 + 0.10636A15 + 0.08158}
\end{Highlighting}
\end{Shaded}

\emph{2. You are welcome, but not required, to try other (nonlinear)
kernels as well; we're not covering them in this course, but they can
sometimes be useful and might provide better predictions than
vanilladot.}

\emph{3. Using the k-nearest-neighbors classification function kknn
contained in the R kknn package, suggest a good value of k, and show how
well it classifies that data points in the full data set. Don't forget
to scale the data (scale=TRUE in kknn).}

\begin{Shaded}
\begin{Highlighting}[]
\CommentTok{\#load package }
\FunctionTok{library}\NormalTok{(kknn)}

\NormalTok{kknn\_accuracy\_test }\OtherTok{=} \ControlFlowTok{function}\NormalTok{(Z)\{}
\NormalTok{  Pred\_kknn }\OtherTok{\textless{}{-}} \FunctionTok{rep}\NormalTok{(}\DecValTok{0}\NormalTok{,}\FunctionTok{nrow}\NormalTok{(data))}
  \ControlFlowTok{for}\NormalTok{ (i }\ControlFlowTok{in} \DecValTok{1}\SpecialCharTok{:}\FunctionTok{nrow}\NormalTok{(data))\{}
    
    \CommentTok{\#model creation using scaled data; ensuring it doesnt use i itself }
\NormalTok{    kknn\_model }\OtherTok{\textless{}{-}} \FunctionTok{kknn}\NormalTok{(R1}\SpecialCharTok{\textasciitilde{}}\NormalTok{A1}\SpecialCharTok{+}\NormalTok{A2}\SpecialCharTok{+}\NormalTok{A3}\SpecialCharTok{+}\NormalTok{A8}\SpecialCharTok{+}\NormalTok{A9}\SpecialCharTok{+}\NormalTok{A10}\SpecialCharTok{+}\NormalTok{A11}\SpecialCharTok{+}\NormalTok{A12}\SpecialCharTok{+}\NormalTok{A14}\SpecialCharTok{+}\NormalTok{A15, data[}\SpecialCharTok{{-}}\NormalTok{i,], data[i,], }\AttributeTok{k =}\NormalTok{ Z, }\AttributeTok{scale =} \ConstantTok{TRUE}\NormalTok{)}
    
    \CommentTok{\#to round values}
\NormalTok{    Pred\_kknn[i] }\OtherTok{\textless{}{-}} \FunctionTok{as.integer}\NormalTok{(}\FunctionTok{fitted}\NormalTok{(kknn\_model) }\SpecialCharTok{+} \FloatTok{0.5}\NormalTok{)}
\NormalTok{  \}}
  
  \CommentTok{\#accuracy calculation}
\NormalTok{  accuracy\_out }\OtherTok{\textless{}{-}} \FunctionTok{sum}\NormalTok{(Pred\_kknn }\SpecialCharTok{==}\NormalTok{ data[,}\DecValTok{11}\NormalTok{]) }\SpecialCharTok{/} \FunctionTok{nrow}\NormalTok{(data)}
  
  \FunctionTok{return}\NormalTok{(accuracy\_out)}
\NormalTok{\}}
\NormalTok{acc }\OtherTok{\textless{}{-}} \FunctionTok{rep}\NormalTok{(}\DecValTok{0}\NormalTok{,}\DecValTok{20}\NormalTok{) }
\ControlFlowTok{for}\NormalTok{ (Z }\ControlFlowTok{in} \DecValTok{1}\SpecialCharTok{:}\DecValTok{20}\NormalTok{)\{}
\NormalTok{  acc[Z] }\OtherTok{=} \FunctionTok{kknn\_accuracy\_test}\NormalTok{(Z)}
\NormalTok{\}}

\CommentTok{\#accuracy percentage }
\NormalTok{kknn\_acc }\OtherTok{=} \FunctionTok{as.matrix}\NormalTok{(acc }\SpecialCharTok{*} \DecValTok{100}\NormalTok{)}

\NormalTok{kknn\_acc}
\end{Highlighting}
\end{Shaded}

\begin{verbatim}
##           [,1]
##  [1,] 81.49847
##  [2,] 81.49847
##  [3,] 81.49847
##  [4,] 81.49847
##  [5,] 85.16820
##  [6,] 84.55657
##  [7,] 84.70948
##  [8,] 84.86239
##  [9,] 84.70948
## [10,] 85.01529
## [11,] 85.16820
## [12,] 85.32110
## [13,] 85.16820
## [14,] 85.16820
## [15,] 85.32110
## [16,] 85.16820
## [17,] 85.16820
## [18,] 85.16820
## [19,] 85.01529
## [20,] 85.01529
\end{verbatim}

\begin{Shaded}
\begin{Highlighting}[]
\CommentTok{\#maximum accuracy is 85.321\%}
\CommentTok{\#12 and 15 have the highest accuracy.}
\end{Highlighting}
\end{Shaded}


\end{document}
